% AER-Article.tex for AEA last revised 22 June 2011
\documentclass[AER]{AEA}

%%%%%% NOTE FROM OVERLEAF: The mathtime package is no longer publicly available nor distributed. We recommend using a different font package e.g. mathptmx if you'd like to use a Times font.
% \usepackage{mathptmx}

% The mathtime package uses a Times font instead of Computer Modern.
% Uncomment the line below if you wish to use the mathtime package:
%\usepackage[cmbold]{mathtime}
% Note that miktex, by default, configures the mathtime package to use commercial fonts
% which you may not have. If you would like to use mathtime but you are seeing error
% messages about missing fonts (mtex.pfb, mtsy.pfb, or rmtmi.pfb) then please see
% the technical support document at http://www.aeaweb.org/templates/technical_support.pdf
% for instructions on fixing this problem.

% Note: you may use either harvard or natbib (but not both) to provide a wider
% variety of citation commands than latex supports natively. See below.

% Uncomment the next line to use the natbib package with bibtex 
%\usepackage{natbib}

% Uncomment the next line to use the harvard package with bibtex
%\usepackage[abbr]{harvard}

% This command determines the leading (vertical space between lines) in draft mode
% with 1.5 corresponding to "double" spacing.

\draftSpacing{1.5}

\begin{document}

\title{How does Single Parent Family Affect Children's Human Capital Investment}
\shortTitle{SP family effect}
\author{Yi Jie Wu, Xiang Jyun Jhang\thanks{Wu: Department of Economics, National Taiwan University, b08302129@ntu.edu.tw; Jhang: Department of Economics, National Taiwan University, b08303024@ntu.edu.tw.  We are sincerely grateful for comments and advices from prof. Tzu-Ting Yang}}
\date{\today}
\pubMonth{JULY}
\pubYear{2023}
\JEL{}
\Keywords{}

\begin{abstract}
    We estimate the impact of living in the single parent family on the children's future educational attainment by using a comprehensive panel survey data in Taiwan (TEPS).  By conducting PDS method, it's found that the experience of living under single parent family will decrease the probability of attaining bachelor degree by  percent, and master degree by  percents also.  We also finds that there is a gap between the probability for two different groups of sample, which we suggest is resulted from the Taiwan Educational Reform in 2000s.
\end{abstract}


\maketitle

American Economic Review Pointers:

\begin{itemize}
\item Do not use an ``Introduction'' heading. Begin your introductory material
before the first section heading.

\item Avoid style markup (except sparingly for emphasis).

\item Avoid using explicit vertical or horizontal space.

\item Captions are short and go below figures but above tables.

\item The tablenotes or figurenotes environments may be used below tables
or figures, respectively, as demonstrated below.

\item If you have difficulties with the mathtime package, adjust the package
options appropriately for your platform. If you can't get it to work, just
remove the package or see our technical support document online (please
refer to the author instructions).

\item If you are using an appendix, it goes last, after the bibliography.
Use regular section headings to make the appendix headings.

\item If you are not using an appendix, you may delete the appendix command
and sample appendix section heading.

\item Either the natbib package or the harvard package may be used with bibtex.
To include one of these packages, uncomment the appropriate usepackage command
above. Note: you can't use both packages at once or compile-time errors will result.

\end{itemize}

\section{Data and Sample} % first paragraph: how the sample is built from the dataset?

    The dataset we use are Taiwan Education Panel Survey (TEPS) and Taiwan Education Panel Survey and Beyond (TEPS-B), in which two group of children are being surveyed consistently until 2020. The former dataset contains the in-school performance and characteristics of each children by surveying on their parents, teachers and themselves. The latter dataset contains the characteristics after the children entered into the labor market, which usually heppens when they age above 23.  In the whole dataset, there are two main group of sample, \textit{\textbf{senior high (SH)}} group and \textit{\textbf{core population (CP)}} group.  The only difference between these two groups is the born year: the born year for SH group is 1984-1985, but for CP group is 1988-1989.

    Since our goal is to estimate the causal effect of living in single parent family on children's future educational attainment, we first start on defining \textbf{living in single parent family}.  The treatment variable is proposed so as to help us to indicate whether the children is under single parent family in his school age:

    \[
        \textit{SP}_i =
        \left\{\begin{aligned}
        &1,\quad\text{if child $i$ under SP family in junior/senior high} \\
        &0,\quad\text{o.w}
        \end{aligned}\right.
    \]

    

\section{empirical Method} % Second paragraph: what's the assumption and the specification?




\section{Result}


\section{Discussion and Conclusion}



References here (manual or bibTeX). If you are using bibTeX, add your bib file 
name in place of BibFile in the bibliography command.
% Remove or comment out the next two lines if you are not using bibtex.
\bibliographystyle{aea}
\bibliography{BibFile}

% The appendix command is issued once, prior to all appendices, if any.
\appendix

\end{document}

