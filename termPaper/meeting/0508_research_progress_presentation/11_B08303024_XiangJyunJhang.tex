\documentclass{beamer}
\usetheme{Boadilla}
\usecolortheme{spruce}

\title{How does Single Parent Family Affect Children's Human Capital Investment}
\author[Yi Jie Wu, Xiang Jyun Jhang]{Yi Jie Wu\inst{1} \and Xiang Jyun Jhang\inst{2}}
\institute[NTU]
{
    \inst{1}
    Department of Economics \\
    National Taiwan University
    \and
    \inst{2}
    Department of Economics \\ 
    National Taiwan University
}
\date{2023/05/08}

\begin{document}
\frame{\titlepage}


\begin{frame} % 2.
\frametitle{Motivation}
\begin{itemize}
    \item Single parent family might fail to provide children with stable environment for learning
    \begin{itemize}
        \item The economic condition for single parent family is usually more fragile than traditional nuclear family
        \item Besides, single parent might not give enough mental support to children
    \end{itemize}
    \item We aim to estimate the negative impact on children's education attainment
\end{itemize}
\end{frame}


\begin{frame} % 3.
\frametitle{Dataset}
\framesubtitle{Taiwan Education Panel Survey and Beyond, SRDA}
\begin{itemize}
    \item A panel data tracking down two different groups of children across almost 20 years
    \begin{itemize}
        \item group 1: born in 1984-1985
        \item group 2: born in 1988-1989
    \end{itemize}
    \item Survey on the children, their parent and teachers, even after they enter labor market
    \item So we have comprehensive information about the children's behavior, relationship and academic performance
\end{itemize}
\end{frame}


\begin{frame} % 4.
\frametitle{Sample}
\begin{itemize}
    \item Treatment variable ($D_i$): single parent family
    \item Outcome variable ($Y_i$): university degree, public university and working years
    \item In order to find possbile confounders, we consider the covariates from these survey data for each group:
    \begin{itemize}
        \item student
        \item classroom teacher
        \item Chinese, English, Math teacher
        \item parents
    \end{itemize}
    \item We pick an abundance of covariates to conduct PDS
\end{itemize}
\end{frame}


\begin{frame} % 5.
\frametitle{Empirical Specification}
\framesubtitle{Post-Double Selection}
\begin{itemize}
    \item With PDS, we can identify the confounders in roughly 110 selected covariates and identify the causal relationship
    \[
        Y_i = \beta_0 + \beta_1 D_i + \bigcup_{j \in A \cup B} \pi_j W_i^j + \epsilon_i
    \]
    where $A,B$ are the Lasso-selected covariates at step 1 \& 2 in PDS.
    \item In later analysis, we try to compare if there is any difference of effect between groups
\end{itemize}
\end{frame}
    


\end{document}
